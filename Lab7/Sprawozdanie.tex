\documentclass{article}
\usepackage{graphicx}
\usepackage{float}
\usepackage{titlesec}
\usepackage{datetime}
\usepackage{geometry}
\usepackage{minted}
\usepackage{xcolor}
\usepackage{listings}
\usepackage{caption}
\usepackage{gensymb}
\usepackage[document]{ragged2e}
\usepackage[hidelinks]{hyperref}
\usepackage{booktabs}
\usepackage{placeins}

\geometry{
 a4paper,
 left=25mm,
 top=25mm,
 }
\captionsetup{hypcap=false}
\newdateformat{daymonthyear}{\THEDAY .\THEMONTH .\THEYEAR}
\title{
  \centering
  \includegraphics[width=\textwidth]{src/images/logo_PWr_kolor_poziom.png}\\
  \fontsize{28pt}{30pt}\selectfont Sprawozdanie 7\\
  \fontsize{14pt}{30pt}\selectfont Ćwiczenie 7.Modemy.Transmisja sygnałów cyfrowych}
\author{Krzysztof Zalewa}
\date{\daymonthyear\today}

\renewcommand*\contentsname{Spis treści}
\renewcommand{\listingscaption}{Fragment kodu}
\begin{document}
  \maketitle
  \pagebreak
  \tableofcontents
  \pagebreak
  \section{Wstęp teoretyczny}
    \subsection{Systemy nawigacji satelitarnej}
    \subsubsection{Budowa i zasada działania systemu GPS}
    \begin{figure}[ht]
      \centering
      \includegraphics[width=\textwidth]{src/images/logo_PWr_kolor_poziom.png}
      \caption{Schemat działania systemów stadiometrycznych}
      \label{fig:gps}
    \end{figure}
    
  \raggedright
  \section{Zadanie laboratoryjne}
    \subsection{Treść zadania}
    
    \subsection{Opis działania programu}
    
    \subsection{Kod programu}
        \begin{frame}
            \scriptsize
            \inputminted[
                style={vs},
                breaklines,
                breakanywhere, 
                linenos, 
                tabsize=4 
            ]{python}{Lab5.py}
            \vspace{1em}
            \captionsetup{justification=centering}
            \captionof{listing}{Kodu programu}
            \label{lst:code}
        \end{frame}      
  \section{Wnioski}

  \section{Źródła}
  \begin{itemize}
    \item 
  \end{itemize}
\end{document}