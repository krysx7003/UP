\documentclass{article}
\usepackage[T1]{fontenc}
\usepackage{graphicx}
\usepackage{datetime}
\usepackage{geometry}
\usepackage{placeins}
\usepackage{minted}
\usepackage{xcolor}
\usepackage{caption}
\usepackage{lmodern} 
\usepackage[document]{ragged2e}
\usepackage[hidelinks]{hyperref}
\usepackage{enumitem}
\geometry{
 a4paper,
 left=25mm,
 top=25mm,
 }
\captionsetup{hypcap=false} 
\newdateformat{daymonthyear}{\THEDAY .\THEMONTH .\THEYEAR}
\title{
  \centering
  \includegraphics[width=\textwidth]{images/logo_PWr_kolor_poziom.png}\\
  \fontsize{28pt}{30pt}\selectfont Sprawozdanie 7\\
  \fontsize{14pt}{30pt}\selectfont Ćwiczenie 7.Modemy. Transmisja sygnałów cyfrowych}
  \author{Krzysztof Zalewa,Wiktor Wojnar}
\date{\daymonthyear\today}
\renewcommand*\contentsname{Spis treści}
\renewcommand{\figurename}{Rysunek}
\renewcommand{\listingscaption}{Fragment kodu}
\begin{document}
    \maketitle
    \pagebreak
    \tableofcontents
    \FloatBarrier
    \section{Wstęp teoretyczny}
        \subsection{Zasady i typu modulacji}
        \subsection{Standardy modulacji/kodowania}
            \subsubsection{Ethernet}
            \subsubsection{WiFi}
            \subsubsection{CD-DVD}
            \subsubsection{zasady realizacji transmisji}
        \subsection{Metody konfiguracji modemów.}
            \subsubsection{Komendy Hayes}
                Komendy Hayes (zbiór komend AT) to specjalny język komend originalnie stworzony
                na potrzeby modemu firmy Hayes w 1981. Obecnie zbiór ten stał się standardem
                i jest używany w większości nowoczesnych urządzeń. Zbiór ten można podzielić na 
                cztery grupy:
                \begin{enumerate}
                    \item \textbf{Komendy podstawowe} - Duża litera i cyfra.Np I0.
                    \item \textbf{Komendy rozszerzone} - Znak \& ,duża litera i cyfra. Rozszerza podstawowy 
                        zbiór więc I0 != \&I0.
                    \item \textbf{Komendy własne} - Zwykle poprzedzone \textbackslash lub \%. Te komendy są 
                        bardzo różne ponieważ są pozostawione potrzebom producentów.
                    \item \textbf{Komendy rejestrów} - S n gdzie n to numer rejestru. Bezpośrednio modyfikuje
                        miejsce w pamięci urządzenia.
                \end{enumerate}
                \begin{table}
                    \begin{tabular}{|l|l|p{8cm}|}
                        \hline
                        Modem A & Modem B & Komentaż \\ \hline
                        ATDT12345678987 & & Użytkownik A podaje komendę do modemu A ATtention; D-Dial; T-Touch-Tone;
                        Zadzwoń na ten numer 12345678987 \\ \hline
                        & Dzwoni & Modem A rozpoczyna dzwonienie na modem B. Modem B daje znać o przychodzącym 
                        połączniu\\ \hline
                        & ATA & Komputer B odbiera połączenie\\ \hline
                        Połączenie & Połączenie & Oba modemy wyświetlają informację o poprawnym połączeniu.\\ \hline
                        qwerty & qwerty & Kiedy modemy są połączone\\ \hline
                        & +++ & \\ \hline
                        & OK & \\ \hline
                        & ATH & \\ \hline
                        Brak połączenia & OK & \\ \hline
                    \end{tabular}
                \end{table}
            \subsubsection{Tryby pracy/diagram stanów modemu}
        \begin{figure}[ht]
            \centering
            \includegraphics[width=\textwidth]{images/logo_PWr_kolor_poziom.png}
            \caption{}
            \label{fig:tex2}
        \end{figure}
        \FloatBarrier
    \section{Zadanie laboratoryjne}
        \subsection{Treść zadania}
        \subsection{Opis działania programu}
        \subsection{Kod programu}
            \begin{frame}
                \scriptsize
                \inputminted[
                    style={vs},
                    breaklines,
                    breakanywhere, 
                    linenos, 
                    tabsize=4 
                ]{c++}{Lab7.cpp}
                \vspace{1em}
                \captionof{listing}{Fragment kodu z programu}
                \label{lst:code}
            \end{frame}
    \section{Wnioski}
    \section{Źródła}
        \begin{enumerate}[label=\arabic*.]
            \item \url{https://en.wikipedia.org/wiki/Hayes_AT_command_set}
        \end{enumerate}
\end{document}